% Author : Frédéric CHEVALIER
% Created in: 2017-02-19
% Modified in:


%%%%%%%%%%%%%%%%%%%%%%%%%%%% Document Setup %%%%%%%%%%%%%%%%%%%%%%%%%%%%

%\documentclass[11pt,a4paper]{report}
\documentclass[11pt]{report}


% Encoding of the tex file
\usepackage{cmap}		% Needed for correct copy-paste from pdf file
\usepackage[utf8]{inputenc}
\usepackage[T1]{fontenc}


% Language parameters
\usepackage[english]{babel}


% Page setup
%\usepackage{xspace}	% spaces between quotes
\usepackage{setspace}	% line spacing (\onehalfspacing = 1.5 ; \doublespacing = 2)
\onehalfspacing
\usepackage[paper=letterpaper,
	%includefoot, 				% Uncomment to put page number above margin
	marginparwidth=30.5mm,   	% Length of section titles
	%marginparsep=1.5mm,      	% Space between titles and text
	margin=20mm,             	% 25mm margins
	%includemp]
]{geometry}

\usepackage[usenames,dvipsnames]{color}
\usepackage{hyperref}
\definecolor{darkblue}{rgb}{0.0,0.0,0.3}

\hypersetup{
	colorlinks,
	breaklinks,
	hypertexnames=false,				% needed when resetting counter to have toc link pointing on the right section
    linkcolor=darkblue,
    urlcolor=darkblue,
    anchorcolor=darkblue,
    citecolor=darkblue,
    pdftitle={Publish fastq manual},    % title
    pdfauthor={Frédéric CHEVALIER},
    pdfkeywords={},
}

\usepackage{enumitem}


% Characters and formating
\usepackage{amsmath}
\usepackage{amsfonts}
\usepackage{amssymb}
\usepackage{listings}	% verbatim environment with newline possibilities
\lstset{
	language=bash,					% the language of the code
	tabsize=4,						% size of the tabulation
	basicstyle=\singlespace\footnotesize\ttfamily,	% the size of the fonts that are used for the code
	frame=single,					% adds a frame around the code
	breaklines=true,				% sets automatic line breaking
	showstringspaces=false,			% underline spaces within strings only
	columns=flexible,
	%numbers=left,
	%numberstyle=\tiny \bf,
	%numberfirstline=true,
	commentstyle=\color{blue},
	keywordstyle=\color{Maroon},
	numberstyle=\tiny\color{mygray},	% the style that is used for the line-numbers
	literate={à}{{\`a}}1 {è}{{\`e}}1 {é}{{\'e}}1
}
\usepackage{footmisc}


% Graphics
\usepackage{graphicx}
\usepackage{wasysym}    % ajout de symboles (ex: symbol diamètre)


% Bibliography
\usepackage[sort]{natbib}				% multiple citations; option [round] if brakets needed without bibpunct package
\bibpunct[; ]{(}{)}{~;}{a}{}{,}			% citation formating within the text
\usepackage[notbib,nottoc,section]{tocbibind}	% to avoid lof / lot inclusion problem (conflict with hyperref package)



%%%%%%%%%%%%%%%%%%%%%%%% End Document Setup %%%%%%%%%%%%%%%%%%%%%%%%%%%%


%%%%%%%%%%%%%%%%%%%%%%%%%%% Helper Commands %%%%%%%%%%%%%%%%%%%%%%%%%%%%

%% A utiliser avec les packages verbatim ou morverb pour effectuer des retour à la ligne (mais ne fait pas de césure)
%\makeatletter
%\def\@xobeysp{ }
%\makeatother

% Command for vertical spacing in title page
\newcommand{\HRule}{\rule{\linewidth}{0.5mm}}

% Command for blank page insertion
\newcommand{\blankpage}[1][empty]{
    \newpage
    \thispagestyle{#1}
    \mbox{}
    \newpage
}

\renewcommand{\thesection}{\arabic{section}.}



%%%%%%%%%%%%%%%%%%%%%%%% End Helper Commands %%%%%%%%%%%%%%%%%%%%%%%%%%%


%%%%%%%%%%%%%%%%%%%%%%%%% Begin Document %%%%%%%%%%%%%%%%%%%%%%%%%%%%

\begin{document}

%\maketitle
\input{"Title page.tex"}
\input{"Version.tex"}

\tableofcontents



%========================
 \chapter*{Introduction}
%========================
\addcontentsline{toc}{chapter}{Introduction}

This script aims to reorganize fastq files generated by the Illumina bcl2fq tool in a standardized directory named after the flow cell ID and containing sub-directories sorted either by samples or by projects. This script was written to replace Juan Peralta's scripts which were used to reorganize files generated from the Illumina Casava pipeline in the same sample or project way. However Juan's scripts are no longer working because bcl2fq does not organize the data the same way the Casava pipeline did. Therefore a complete rewriting was needed.

The bcl2fq tool demultiplexes and converts raw data from bcl files (acquisition files) to fastq files. This tool comes in replacement of the no longer supported Casava pipeline.\footnote{\url{https://support.illumina.com/sequencing/sequencing_software/casava.html}} This pipeline used to export data by storing fastq files in project and samples directories. Once the Casava output folder uploaded on the ranch, Juan's scripts were used to create a folder named after the flow cell ID and then publish the uploaded data in this folder by reorganizing the Casava output based on lanes and samples. Some users have written scripts based on this folder structure to process data. But the output folder generated by bcl2fq is much simpler than the one generated by the Casava pipeline: all data are stored in a unique directory or in sub-directories corresponding to each project specified in the sample sheet. The present script aims to recreate the publication style generated by Juan's scripts.

This documentation details the different options available from this script and basic usage examples. Source code of the script can be found at the end.



%==================================
 \chapter*{Installation and usage}
%==================================
\addcontentsline{toc}{chapter}{Installation and usage}
\setcounter{section}{0}

%-----------------------
 \section{Installation}
%-----------------------

The publish-fq.sh script comes with an optional email template. The script can be located anywhere on the server use to publish the fastq files. For the email option to work, both files need to be at the same location, ideally in a folder. The script needs to be executable: the file permission should be \lstinline[basicstyle=\ttfamily]{rwxr--r--}. If it is not the case, the command \lstinline[basicstyle=\ttfamily]{chmod u+rwx,go+r publish-fq.sh} needs to be executed at the location of the script. For accessing the script from everywhere, the folder path of the script can be append to the PATH environment variable. To do so in a bash environment, be at the location of the script, execute the command \lstinline[basicstyle=\ttfamily]{pwd}, copy the path, open the \$HOME/.bashrc file and paste \lstinline[basicstyle=\ttfamily]{export PATH=\$PATH:script_folder} where \lstinline[basicstyle=\ttfamily]{script_folder} is the path given by \lstinline[basicstyle=\ttfamily]{pwd}. If you are using another shell than bash, this procedure have to be adapted.


%----------------
 \section{Usage}
%----------------

Once the publish-fq.sh installed, a usage message is accessible by executing the script without arguments or with -h or --help arguments. The usage message details all options available:

\begin{lstlisting}[language=]
     publish-fq.sh  -d|--dir bcl2fq_dir -s|--ss samplesheet -o|--dest destination_dir -p|--pjt -l|--spl -e|--em user_list -h|--help

Aim: Publish fastq files the "old Casava way" (by samples or by projects).

Version: 0.1

Options:
    -d, --dir       path to the output directory of bcl2fq containing the fastq files, the reports and stats folders
    -s, --ss        path to the samplesheet
    -o, --dest      path to the destination directory [default: /data/HiSeq]
    -p, --pjt       by project publishing [Incompatible with -l]
    -l, --spl       by sample publishing [Incompatible with -p] [default]
    -e, --em        list of user to which send an email when publishing is done
                        The list must be space separated (eg, -e brad janet) [default: janet]
    -h, --help
\end{lstlisting}


Here are the details of the options:
\begin{description} [labelindent=2cm]
	\item [\texttt{\textbf{-{}-dir}}:] This option is mandatory. The path to the bcl2fq output folder must be given. This path can point anywhere on the server.
	\item [\texttt{-{}-ss:}] This option is mandatory. The path to the samplesheet used with bcl2fq must be given. The samplesheet needs to be uploaded on the server (in the folder of the bcl2fq folder for instance). This path can point anywhere on the server.
	\item [\texttt{-{}-dest:}] The path to the destitination (output) folder where data will be moved and renamed to mimic the "old Casava way". This path can point anywhere on the server but the /data/HiSeq folder is set by default.
	\item [\texttt{-{}-pjt:}] This option acts as a switch to publish the data following the project names present in the samplesheet. Cannot be used with \texttt{-{}-spl}.
	\item [\texttt{-{}-spl:}] This option acts as a switch to publish the data following the lanes and sample names present in the samplesheet. Cannot be used with \texttt{-{}-pjt}.
	\item [\texttt{-{}-em:}] This option allows people to be informed by email at the end of the publishing. Roy Garcia is informed by default but other people can be included. The email user name must be used and can be followed by @txbiomed.org but this is not mandatory. Email addresses outside of txbiomed.org domain does not work. If the script does not find any email template, no email is sent and an error message is issued. % Email addresses outside of txbiomed.org domain may work (but in this case a complete email address must be entered).
\end{description}


%--------------------
 \section{Operation}
%--------------------

When a sequencing run is done and bcl2fq has converted the bcl files into fastq files, the generated output folder needs to be uploaded (using rsync or scp) on one of our ranch servers to be published (bulmer, mendel, etc.). Once on the server, the publish script can be called to reorganize the data. During the process, several explicit messages will appear:
\begin{itemize}
	\item info message in green about current process,
	\item warning message in yellow about unexpected but not fatal problem,
	\item error message in red about unexpected and fatal problem.
\end{itemize}

When process is done, the final step consists in sending an email at least to the operator (Roy Garcia) and optionally to the PI or the designated correspondent. This email inform people on where the data are stored and send the sequencing report to these persons. To work properly, a file named \verb|email_template| needs to be present along the script. If not, no email will be sent despite the activation of the option. The report will be compressed in tar.gz file. This report archive will actually contains the Reports and the Stats folders generated by bcl2fq. Presence of each folder will be check and warning message will be issued specifically if nothing is found.
~\\

\noindent \textbf{N.B.}: if publish-fastq.sh is interrupted in the middle of the process, data will be corrupted. In this case, the bcl2fq output folder and the publish folder must be deleted and the bcl2fq output folder must be uploaded again on the server. Then the publishing can be restarted.


%------------------------
 \section{Usage example}
%------------------------

Here is an example of how to use the script:
\begin{lstlisting}[language=bash]
# A folder named shsmchronobio contains the output of the bcl2fq and was uploaded
# to the HiSeq folder on the ranch
cd /data/HiSeq/
ls

# In the case publish-fq.sh can be reached through PATH
# The following command will publish data contained in the shsmchronobio using the 
# samplesheet present in the folder and sending to the corresponding author
publish-fq.sh -d shsmchronobio -s shsmchronobio/samplesheet.csv -e brad

# In the case publish-fq.sh cannot be reached through PATH
# The following example does the same as previously. Notice the change at the beginning:
# path/to/ is the absolute path of the folder containing publish-fq.sh
/path/to/publish-fq.sh -d shsmchronobio -s shsmchronobio/samplesheet.csv -e brad
\end{lstlisting}
~\\


%=======================
 \chapter*{Source code}
%=======================
\addcontentsline{toc}{chapter}{Script code}
\setcounter{section}{0}

%----------------------
 \section{Script code}
%----------------------

This is the code of the publish-fq.sh script. The version of the script is indicated within the code and details about version history is given in the corresponding section of the script.
\lstinputlisting{"../publish-fq.sh"}
~\\

%-------------------------
 \section{Email template}
%-------------------------

This is the email template called by the script in order to send an email when the publishing is done. The body of the template can be adjusted according to any specific needs. This template contains two variables:
\begin{itemize}
	\item \verb|${i%%@*}|: this variable contains the email user name of the recipient,
	\item \verb|$dir_out|: this variable contains the path to the published data.
\end{itemize}

% Uncomment if running on Linux
%\lstinputlisting[language=]{"../email_template"}
% Uncomment if running on Windows
\lstinputlisting[language=]{"../email_template."}
~\\

~\\


\end{document}

%%%%%%%%%%%%%%%%%%%%%%%%%% End Document %%%%%%%%%%%%%%%%%%%%%%%%%%%%%
