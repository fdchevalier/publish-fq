% Author : Frédéric CHEVALIER
% Created in: 2017-02-19
% Modified in: 2017-05-03


%%%%%%%%%%%%%%%%%%%%%%%%%%%% Document Setup %%%%%%%%%%%%%%%%%%%%%%%%%%%%

%\documentclass[11pt,a4paper]{report}
\documentclass[11pt]{report}


% Encoding of the tex file
\usepackage{cmap}		% Needed for correct copy-paste from pdf file
\usepackage[utf8]{inputenc}
\usepackage[T1]{fontenc}


% Language parameters
\usepackage[english]{babel}


% Page setup
\usepackage{setspace}	% line spacing (\onehalfspacing = 1.5 ; \doublespacing = 2)
\onehalfspacing
\usepackage[paper=letterpaper,
	%includefoot, 				% Uncomment to put page number above margin
	marginparwidth=30.5mm,   	% Length of section titles
	%marginparsep=1.5mm,      	% Space between titles and text
	margin=20mm,             	% 25mm margins
	%includemp]
]{geometry}

\usepackage[usenames,dvipsnames]{color}
\usepackage{hyperref}
\definecolor{darkblue}{rgb}{0.0,0.0,0.3}

\hypersetup{
	colorlinks,
	breaklinks,
	hypertexnames=false,				% needed when resetting counter to have toc link pointing on the right section
    linkcolor=darkblue,
    urlcolor=darkblue,
    anchorcolor=darkblue,
    citecolor=darkblue,
    pdftitle={Publish fastq manual},    % title
    pdfauthor={Frédéric CHEVALIER},
    pdfkeywords={},
}

\usepackage{enumitem}


% Characters and formating
\usepackage{amsmath}
\usepackage{amsfonts}
\usepackage{amssymb}
\usepackage{listings}	% verbatim environment with newline possibilities
\lstset{
	language=bash,					% the language of the code
	tabsize=4,						% size of the tabulation
	basicstyle=\singlespace\footnotesize\ttfamily,	% the size of the fonts that are used for the code
	frame=single,					% adds a frame around the code
	breaklines=true,				% sets automatic line breaking
	showstringspaces=false,			% underline spaces within strings only
	columns=flexible,
	%numbers=left,
	%numberstyle=\tiny \bf,
	%numberfirstline=true,
	commentstyle=\color{blue},
	keywordstyle=\color{Maroon},
	numberstyle=\tiny\color{mygray},	% the style that is used for the line-numbers
	literate={à}{{\`a}}1 {è}{{\`e}}1 {é}{{\'e}}1
}
\usepackage{footmisc}
\usepackage{tabularx, booktabs, makecell}	% table formating
%\usepackage{hyphenat}


% Graphics
\usepackage{graphicx}
\usepackage{wasysym}    % ajout de symboles (ex: symbol diamètre)


% Bibliography
\usepackage[sort]{natbib}				% multiple citations; option [round] if brakets needed without bibpunct package
\bibpunct[; ]{(}{)}{~;}{a}{}{,}			% citation formating within the text
\usepackage[notbib,nottoc,section]{tocbibind}	% to avoid lof / lot inclusion problem (conflict with hyperref package)



%%%%%%%%%%%%%%%%%%%%%%%% End Document Setup %%%%%%%%%%%%%%%%%%%%%%%%%%%%


%%%%%%%%%%%%%%%%%%%%%%%%%%% Helper Commands %%%%%%%%%%%%%%%%%%%%%%%%%%%%

%% A utiliser avec les packages verbatim ou morverb pour effectuer des retour à la ligne (mais ne fait pas de césure)
%\makeatletter
%\def\@xobeysp{ }
%\makeatother

% Command for vertical spacing in title page
\newcommand{\HRule}{\rule{\linewidth}{0.5mm}}

% Command for blank page insertion
\newcommand{\blankpage}[1][empty]{
    \newpage
    \thispagestyle{#1}
    \mbox{}
    \newpage
}

\renewcommand{\thesection}{\arabic{section}.}



%%%%%%%%%%%%%%%%%%%%%%%% End Helper Commands %%%%%%%%%%%%%%%%%%%%%%%%%%%


%%%%%%%%%%%%%%%%%%%%%%%%% Begin Document %%%%%%%%%%%%%%%%%%%%%%%%%%%%

\begin{document}

%\maketitle
\input{"Title page.tex"}
\input{"Version.tex"}

\tableofcontents



%========================
 \chapter*{Introduction}
%========================
\addcontentsline{toc}{chapter}{Introduction}

This script aims at reorganizing fastq files generated by the Illumina bcl2fastq tool in a standardized fashion with a directory named after the flow cell ID containing sub-directories sorted either by lanes/samples or by projects. This script comes in replacement of Juan Peralta's scripts which were used to perform the same reorganization on files generated by the Illumina Casava pipeline. But Illumina is pushing for a switch from the old Casava pipeline to the standalone bcl2fastq program. Because bcl2fastq does not organize the data the same way the pipeline did, Juan's scripts are no longer working. Therefore a complete rewriting was needed. Using this opportunity, the script brings also two new functionalities: sending emails to designated persons when data are available and keeping sequences from undetermined indices.

The bcl2fastq tool demultiplexes and converts raw data from bcl files (acquisition files) to fastq files. This tool comes in replacement of the no longer supported Casava pipeline.\footnote{\url{https://support.illumina.com/sequencing/sequencing_software/casava.html}} This pipeline used to export data by storing fastq files in project and samples directories. Once the Casava output folder generated, Juan's scripts were used to create a folder named after the flow cell ID and then to publish the data in this folder by reorganizing the Casava output based on lanes and samples. Some users wrote scripts based on this folder structure to process data. But the output folder generated by bcl2fastq is much simpler than the one generated by the Casava pipeline: all data are stored in a unique directory or in sub-directories corresponding to each project specified in the sample sheet. The present script aims to recreate the publication style generated by Juan's scripts.

This documentation details the different options available from this script and basic usage examples. Source code of the script can be found at the end as well as on \href{https://github.com/fdchevalier/publish-fq}{github}.



%==================================
 \chapter*{Installation and usage}
%==================================
\addcontentsline{toc}{chapter}{Installation and usage}
\setcounter{section}{0}

%-----------------------
 \section{Installation}
%-----------------------

The publish-fq.sh script comes with an optional email template. The script can be located anywhere on the server use to publish the fastq files. For the email option to work, both script and template need to be at the same location, ideally in a folder. The script needs to be executable: the file permission should be \lstinline[basicstyle=\ttfamily]{rwxr--r--}. If it is not the case, the command \lstinline[basicstyle=\ttfamily]{chmod u+rwx,go+r publish-fq.sh} needs to be executed at the location of the script. For accessing the script from everywhere, the folder path of the script can be appended to the PATH environment variable. To do so in a bash environment, be at the location of the script, execute the command \lstinline[basicstyle=\ttfamily]{pwd}, copy the path, open the \$HOME/.bashrc file and paste \lstinline[basicstyle=\ttfamily]{export PATH=$PATH:script_folder} where \lstinline[basicstyle=\ttfamily]{script_folder} is the path given by \lstinline[basicstyle=\ttfamily]{pwd}. If you use another shell than bash, this procedure has to be adapted.


%----------------
 \section{Usage}
%----------------

Once the publish-fq.sh installed, a usage message is accessible by executing the script without arguments or with -h or -{}-help arguments. The usage message details all options available:

\begin{lstlisting}[language=]
     publish-fq.sh  -d|--dir bcl2fq_dir -s|--ss samplesheet -o|--dest destination_dir -a|--app-fid string -k|--keep-und -p|--pjt -l|--spl -e|--em user_list -h|--help

Aim: Publish fastq files the "old Casava way" (by samples or by projects).

Version: 1.0

Options:
    -d, --dir       path to the output directory of bcl2fq containing the fastq files, the reports and stats folders
    -s, --ss        path to the samplesheet
    -o, --dest      path to the destination directory [default: MYPATH]
    -a, --app-fid   text appended to the flow cell ID
    -k, --keep-und  keep undetermined indices [default: no]
                        If the "Keep undetermined" field is present in the Header section of the sample sheet, this option is ignored
    -p, --pjt       by project publishing [incompatible with -l]
    -l, --spl       by sample publishing [incompatible with -p] [default]
    -e, --em        list of user to which send an email when publishing is done
                        The list must be space separated (eg, -e brad janet) [default: janet]
                        If the "Emails" field is present in the Header section of the sample sheet, this option is ignored
    -h, --help      this message
\end{lstlisting}


Here are the details of the options:
\begin{description} [labelindent=2cm]
	\item [\texttt{\textbf{-{}-dir}}:] This option is mandatory. The path to the bcl2fastq output folder must be given. This path can point anywhere on the server.
	\item [\texttt{-{}-ss:}] This option is mandatory. The path to the sample sheet used with bcl2fastq must be given. This path can point anywhere on the server.
	\item [\texttt{-{}-dest:}] The path to the destination (output) folder where the flow cell ID folder will be created to receive the published data in order to mimic the "old Casava way". This path can point anywhere on the server but the HiSeq folder is set by default. This folder is located on the Solexa server.
	\item [\texttt{-{}-app-fid:}] Suffix to add to the flow cell ID folder. This can be useful when two independent publishing from the same flow cell needs to be done (publishing data from demultiplexing of 6 and 8 bp indices for instance). The suffix is automatically prepended with an underscore (e.g., FID\_suffix).
	\item [\texttt{-{}-keep-und:}] This option acts as a switch to copy the fastq files containing sequences with undetermined indices. These files are usually not kept because not used in downstream analysis. But if one wants them, the switch can be used to store them in the publish folder. This option can also be set directly in the sample sheet by adding a line in the Header section (see below). If the latter is used, the present option have no effect.
	\item [\texttt{-{}-pjt:}] This option acts as a switch to publish the data following the project names present in the sample sheet. Cannot be used with \texttt{-{}-spl}.
	\item [\texttt{-{}-spl:}] This option acts as a switch to publish the data following the lanes and sample names present in the sample sheet. Cannot be used with \texttt{-{}-pjt}.
	\item [\texttt{-{}-em:}] This option allows designated correspondent to be informed by email at the end of the publishing. Roy Garcia is informed by default but other people can be included. The email addresses must be separated by a white space. If this is a Texas Biomed address, there is no need to append the @txbiomed.org domain. Email addresses outside of txbiomed.org domain works but the domain needs to be present. If the script does not find any email template, no email is sent and an error message is issued. This option can also be set directly in the sample sheet by adding a line in the Header section (see below). If the latter is used, the present option have no effect.
\end{description}


%--------------------
 \section{Operation}
%--------------------

When a sequencing run is done and bcl2fastq has converted the bcl files into fastq files, the publish script can be called to copy and reorganize the data on the Solexa server. During the process, several explicit messages will appear:
\begin{itemize}
	\item info message in green about current process,
	\item warning message in yellow about unexpected but not fatal problem,
	\item error message in red about unexpected and fatal problem.
\end{itemize}

When process is done, the final step consists in sending an email at least to the operator (Roy Garcia) and optionally to any designated correspondents. The email address(es) can be added in the sample sheet (recommended) or enter manually by the operator. This email informs correspondent(s) on where the data are stored and to what project(s) the data are related, and send the sequencing report to these persons. The report will be compressed in tar.gz file. This report archive will actually contains the Reports and the Stats folders generated by bcl2fastq. Presence of each folder will be check and warning message will be issued specifically if they are missing. To work properly, a file named \verb|email_template.txt| needs to be present along the script. If not, no email will be sent despite the activation of the option.
~\\
\noindent \textbf{N.B.}: If publish-fastq.sh is interrupted in the middle of the process, data will be corrupted. In this case, the published folder must be deleted and the publishing restarted.


%-------------------------
 \section{Usage examples}
%-------------------------

Here is an example of how to use the script:
\begin{lstlisting}[language=bash]
# In the case publish-fq.sh can be reached through PATH
# The following command will publish data contained in the shsmchronobio using the
# sample sheet present in the folder and sending to the corresponding author
publish-fq.sh -d shsmchronobio -s shsmchronobio/samplesheet.csv -e brad

# In the case publish-fq.sh cannot be reached through PATH
# The following example does the same as previously. Notice the change at the beginning:
# path/to/publish-fq/ is the absolute path of the folder containing publish-fq.sh
/path/to/publish-fq/publish-fq.sh -d shsmchronobio -s shsmchronobio/samplesheet.csv -e brad
# or
cd /path/to/publish-fq/
./publish-fq.sh -d shsmchronobio -s shsmchronobio/samplesheet.csv -e brad

# If a new publishing needs to be performed after a first publishing, this option -a must be used.
# Here an example where the "new_publishing" suffix will be appended to the directory name
# (i.e., the flow cell ID)
cd /path/to/publish-fq/
./publish-fq.sh -d shsmchronobio -s shsmchronobio/samplesheet.csv -a new_publishing -e brad
\end{lstlisting}
~\\


%==============================================
 \chapter*{Special fields of the sample sheet}
%==============================================
\addcontentsline{toc}{chapter}{Special fields of the sample sheet}
\setcounter{section}{0}

Two new fields can now be added to the Header section of the sample sheet. These fields are not used by bcl2fastq and does not disturbed the demultiplexing. They allow you to interact with two options related to the publishing step: sending emails and keeping undetermined indices sequences. They are highly recommended in order to avoid misunderstanding between users of the sequencing facility and the operator (Roy). For those using the IEM software from Illumina to generate the sample sheet, these fields have to be added manually by opening the file in a spreadsheet program (Excel, Calc, ...) or in a text editor (Notepad, Geany, ...).

%-----------------
 \section{Emails}
%-----------------

A field named "Emails" (or "Email" or "emails" or "email") can be added to trigger the email option of the publishing script. This method is recommended over asking the operator to add email addresses when publishing data on the ranch. Recipients of the email can be the PI, the research assistant who prepared the libraries and the bioinformatician who will process the data for instance.

This option accepts a list of email addresses separated by white space. Texas Biomed email addresses does not need to have their domain (@txbiomed.org) appended, only the user name is needed. For instance, "brad" is enough to send an email to "fcheval@txbiomed.org". Email addresses outside the Texas Biomed domain can be used but in this case the domain (@something.edu) must be present. For an example, see \nameref{sec:ss-example}

The email option triggered through the sample sheet has priority over the email arguments from command line. Therefore one cannot fill the sample sheet then ask the operator to add an email address with the command line. In this case the sample sheet needs to be modified.


%-------------------------------
 \section{Undetermined indices}
%-------------------------------

Sequences with undetermined indices are generated during the demultiplexing process when bcl2fastq cannot identify the index sequence with certainty. By default bcl2fastq allows one mismatch in the index sequence. This undetermined sequences are usually discarded because unassigned. But if for some reason one wants to keep and analyze them, they can be retained when publishing the data and kept in a dedicated folder within the flow cell ID folder.

To keep the undetermined sequences, a field named "Keep undetermined" can be created in the Header section of the sample sheet the same way as the email field. This field accepts as values yes (or y) and no (or n). By default if the field is absent and if the switch is not used in command line, these sequences are discarded. They can only be regenerated by using bcl2fastq again with the same options as for the first demultiplexing. For a usage example, see \nameref{sec:ss-example}

As for the email option, the keep option triggered through the sample sheet has priority over the switch from command line.


%-------------------------------
 \section{Sample sheet example}
%-------------------------------
\label{sec:ss-example}

Here is an example of the Header section of the sample sheete with the two new fields at the end:

\begin{lstlisting}[language=]
[Header]
IEMFileVersion      	4
Investigator Name   	Tim Anderson
Experiment Name     	Transmission
Date                	4/17/2017
Workflow	            GenerateFASTQ
Application 	        HiSeq FASTQ Only
Assay           	    TruSeq LT
Description
Chemistry  		        Default
Emails          	    brad mary pierre xyz@uthscsa.edu
Keep undetermined   	no
\end{lstlisting}

\noindent \textbf{N.B.}: The sample sheet is a comma separated values file. Therefore the presentation above reflects what one can see in a spreadsheet editor but not in the text editor.



%=======================
 \chapter*{Source code}
%=======================
\addcontentsline{toc}{chapter}{Script code}
\setcounter{section}{0}

%----------------------
 \section{Script code}
%----------------------

This is the code of the publish-fq.sh script. The version of the script is indicated within the code and details about version history is given in the corresponding section of the script.
\lstinputlisting{"../publish-fq.sh"}


%-------------------------
 \section{Email template}
%-------------------------

This is the email\_template.txt called by the script in order to send an email when the publishing is done. The body of the template can be adjusted according to any specific needs. This template contains three variables:
\begin{itemize}
	\item \verb|${i%%@*}|: contains the email user names of the recipients,
	\item \verb|$dir_out|: contains the path to the published data,
	\item \verb|$myprojects|: contains the list of projects founds in the sample sheet. If there is no project, the default value is none.
\end{itemize}

\lstinputlisting[language=]{"../email_template.txt"}


\end{document}

%%%%%%%%%%%%%%%%%%%%%%%%%% End Document %%%%%%%%%%%%%%%%%%%%%%%%%%%%%
